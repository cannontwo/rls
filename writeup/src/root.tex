\documentclass{article}
\usepackage[utf8]{inputenc}
\usepackage[margin=0.5in]{geometry}
\usepackage[onehalfspacing]{setspace}
\usepackage{hyperref}
\usepackage{color}
\usepackage{amsmath}
\usepackage{amssymb}
\usepackage{amsfonts}
\usepackage{booktabs}
\usepackage{enumitem}

\newcommand\fixme[2][FIXME]{\textcolor{red}{\textbf{#1:} #2}}
\newcommand\R[2]{\mathbb{R}^{#1\times #2}}
 
\title{A Derivation of Multivariate Recursive Least Squares}
\author{W. Cannon Lewis II\thanks{Rice University Computer Science Department (\href{http://cannontwo.com}{cannontwo.com})}}

\begin{document}

\maketitle

\abstract{
Least squares estimation---also known as linear regression---is one of the
fundamental tools underlying modern data science and machine learning.  Its
typical exposition assumes a fixed dataset which is analyzed as a whole, but
this assumption is violated when data arrives in a stream over time. The least
squares estimate can instead be computed online using an algorithm known as
recursive least squares. In this note we will derive the update equations for
recursive least squares applied to both centered and uncentered data.
Additionally, we will draw connections between practical implementations of
recursive least squares and $l^2$-regularized least squares, which is also
known as ridge regression.
}

\tableofcontents
\newpage

\section{Notation}
\label{sec:notation}
\begin{table}[htb]
  \centering
  \begin{tabular}{c | c | c}
    \toprule 
    \textbf{Symbol}  & \textbf{Space} & \textbf{Meaning} \\
    \midrule
    $n$ & $\mathbb{N}$ & Input feature dimension \\
    $m$ & $\mathbb{N}$ & Output dimension \\
    $\varphi_t$ & $\R{1}{n}$ & Feature vector at time $t$ \\
    $\mu_x(t)$ & $\R{1}{n}$ & Feature mean at time $t$ \\
    $X_t$ & $\R{t}{n}$ & Design matrix of stacked feature vectors from timesteps 0 to $t$ \\
    $y_t$ & $\R{1}{m}$ & Output vector at time $t$ \\
    $\mu_y(t)$ & $\R{1}{n}$ & Output mean at time $t$ \\
    $Y_t$ & $\R{t}{m}$ & Output matrix of stacked output vectors from timesteps 0 to $t$ \\
    $\Theta$ & $\R{n}{m}$ & True linear model coefficients \\
    $\hat\Theta$ & $\R{n}{m}$ & Estimated model coefficients \\
    $P_t$ & $\R{n}{n}$ & Inverse sample covariance matrix \\
    $Q_t$ & $\R{n}{n}$ & Inverse mean-corrected sample covariance matrix \\
    $R_t$ & $\R{n}{n}$ & Rank-2 update to corrected sample covariance matrix \\
    $V_t$ & $\R{2}{n}$ & $\left[\mu_x(t-1)^\top \quad \varphi_t^\top\right]^\top$ \\
    $\bar{1}$ & $\R{t}{1}$ & A vector whose entries are all 1 \\
    \bottomrule
  \end{tabular}
  \vspace{1em}
  \caption{Notation used in this paper}
  \label{table:notation}
\end{table}

In addition to the table above, we use the notation $x^\top$ to represent the transpose of a matrix $x$.

\section{Problem Formulation}
\label{sec:problem}
Let us define a data stream as a function $F(t): \mathbb{N} \rightarrow \R{1}{n} \times \R{1}{m}$ that, for $t \in \mathbb{N}$,
can be defined as 
\begin{equation} 
  F(t) := (\varphi_t, y_t)
\end{equation}
where $\varphi_t$ and $y_t$ are related according to 
\begin{equation}
  y_t = \varphi_t \Theta + \epsilon_t, \quad \mathbb{E}\left[\epsilon_t\right] = 0
\end{equation} 
Note that we assume that the output of the underlying linear model with true
parameters $\Theta$ is corrupted at each time step by some additive, zero-mean
white noise $\epsilon_t$. Since the purpose of this note is not to explore the
statistical properties of recursive least squares, we skip over further
clarification of the properties of this noise; results follow from the standard
statistical analysis of linear regression~\cite{bishop2006pattern}. 

\subsection{Static Least Squares}
If we wait to observe $T \in \mathbb{N}$ time steps of input-output pairs, we can assemble the matrices $X_T$ and $Y_T$ by vertically stacking the $\varphi_t$ and $y_t$ observations for $t = 1,\ldots,T$. Given these matrices, the least squares estimation problem is formulated as
\begin{align}
  \min_{\hat\Theta} &=  \sum_{t=1}^\top ||y_t - \varphi_t\hat\Theta||_2^2\\
                    &=  ||Y_T - X_T\hat\Theta||_2^2
\end{align}
It is well known (see, e.g.,~\cite{bishop2006pattern}) that the solution to
this problem is given by the ``normal equation''
\begin{equation}
  \label{eq:theta_ls}
  \hat\Theta_{LS}(T) := \left(X_T^\top X_T\right)^{-1}X_T^\top Y_T
\end{equation}


\section{Centered Data}
\label{sec:centered}
From Equation~\ref{eq:theta_ls} we can begin to derive the recursive least
squares estimator. It is worth noting at the beginning of this derivation that
we have implicitly assumed that our data is already centered; in other words,
the relationship between $\varphi_t$ and $y_t$ has no offset term, and so our
model $\hat\Theta_{LS}$ will always predict an output vector of all zeros for
an input vector of all zeros. This is a fine assumption when all of the data
has been collected ahead of time, but breaks down when we want to do recursive
least squares because we cannot estimate the means of our inputs and outputs
ahead of time. Since it is easier to derive the recursive least squares
estimator in the centered case than in the uncentered case, we tackle this
limited version first. 

\subsection{Breaking Up the Normal Equation}
We begin by writing out the normal equation as two sums multiplied together
\begin{align}
  \hat\Theta_{LS}(T) &= (X_T^\top X_T)^{-1} X_T^\top Y_T \\
                  &= \left[\sum_{t=1}^T \varphi_t^\top \varphi_t\right]^{-1} \left[\sum_{t=1}^T\varphi_t^\top y_t \right] \label{eq:two_sums}
\end{align}
Let us define the inverse sample covariance matrix $P_T$ to be the left-hand
term in Equation~\ref{eq:two_sums}, so that we then have
\begin{align}
  \label{eq:sample_covar_split}
  P_T^{-1} &= \sum_{t=1}^T \varphi_t^\top \varphi_t \\
           &= P_{T - 1}^{-1} + \varphi_T^\top \varphi_T \label{eq:p_inv_update}\\
  \implies P_{T - 1}^{-1} &= P_T^{-1} - \varphi_T^\top \varphi_T \label{eq:sub_covar}
\end{align}
Similarly, we can break up the right-hand term in Equation~\ref{eq:two_sums}:
\begin{equation}
  \sum_{t=1}^T\varphi_t^\top y_t = \sum_{t=1}^{T-1} \varphi_t^\top y_t + \varphi_T^\top y_T
\end{equation}

\subsection{Deriving the $\hat\Theta$ Update}
Our normal equation is now
\begin{equation}
  \hat\Theta_{LS} = P_T \cdot \left[\sum_{t=1}^{T-1} \varphi_t^\top y_t + \varphi_T^\top y_T
\right]
\end{equation}
Using the definition of $\hat\Theta_{LS}(T - 1)$, we get
\begin{equation}
  \hat\Theta_{LS}(T) = P_T \cdot \left[P_{T-1}^{-1} \hat\Theta_{LS}(T-1) + \varphi_T^\top y_T\right]
\end{equation}
Substituting in Equation~\ref{eq:sub_covar}:
\begin{align}
  \hat\Theta_{LS}(T) &= P_T \cdot \left[(P_{T}^{-1} - \varphi_T^\top \varphi_T) \hat\Theta_{LS}(T-1) + \varphi_T^\top y_T\right] \\
                     &= \hat\Theta_{LS}(T - 1) - P_T \varphi_T^\top \varphi_T \hat\Theta_{LS}(T-1) + P_T \varphi_T^\top y_T \\
                     &= \hat\Theta_{LS}(T - 1) + P_T \varphi_T^\top \left[y_T - \varphi_T\hat\Theta_{LS}(T-1)\right] \label{eq:centered_theta_update}
\end{align}
Note that the last term in Equation~\ref{eq:centered_theta_update} ($y_T -
\varphi_T\hat\Theta_{LS}(T-1)$) is the prediction error of our model at
timestep $T-1$ on the new datum, so the new estimate of $\Theta$ that we get is
the old estimate plus the prediction error on a new datum filtered by $P_T
\varphi_T^\top$. Intuitively, this represents a reweighting of the prediction
error using our existing estimate of the sample covariance, which effectively
rescales the update to $\Theta$ to take into account the scales of the
coordinates of the features.

\subsection{Deriving the $P_T$ Update}
Though Equation~\ref{eq:p_inv_update} gives us a way to update $P_T^{-1}$
easily with each new datum, to recover $P_T$ and update
$\hat\theta_{LS}$ we would need to invert an $n\times n$ matrix at on each time
step. This is not only computationally expensive for all but small values of
$n$, it also introduces the risk of running into floating-point errors if
$P_T^{-1}$ ever becomes ill-conditioned\footnote{These sorts of issues can also
be dealt with using any number of techniques from numerical linear
algebra~\cite{trefethen1997numerical}}.

We can get around these issues by doing away with $P_T^{-1}$ all together and
deriving a direct update for $P_T$. We do this with the Woodbury matrix
identity~\cite{woodbury1950inverting}, also known as the ``matrix inversion
lemma.'' This result tells us that for matrices $A, U, C, V$ such that $UCV$ has rank $k$, the inverse of the rank-$k$ update is given by:
\begin{equation}
  \label{eq:woodbury}
  \left(A + UCV\right)^{-1} = A^{-1} - A^{-1}U\left(C^{-1} + VA^{-1}U\right)^{-1}VA^{-1}
\end{equation}
In our case, since we are doing a rank-1 update $\varphi_T^\top \varphi_T$ to
$P_{T-1}^{-1}$, this lemma gives us that
\begin{equation}
  \label{eq:centered_p_update}
  P_T = \left(P_{T-1}^{-1} + \varphi_T^\top \varphi_T\right)^{-1} = P_{T - 1} - \frac{P_{T-1}\varphi_T^\top \varphi_T P_{T-1}}{1 + \varphi_T P_{T-1} \varphi_T^\top}
\end{equation}
And just like that, we're done! Equations~\ref{eq:centered_theta_update}
and~\ref{eq:centered_p_update} give us the update equations that define the
recursive least squares algorithm. At each timestep $t$, we simply need to:
\begin{enumerate}
\begin{singlespace}
  \item Record $\varphi_t$ and $y_t$ from our datastream $F(t)$. 
  \item Calculate $P_t$ from $P_{t-1}$ and $\varphi_t$ using Equation~\ref{eq:centered_p_update}. 
  \item Calculate $\hat\Theta_{LS}(t)$ from $\hat\Theta_{LS}(t-1)$, $\varphi_t$, $y_t$, and $P_{t}$ using Equation~\ref{eq:centered_theta_update}. 
\end{singlespace}
\end{enumerate}


\section{Uncentered Data}
\label{sec:uncentered}
In the general linear regression setting, we cannot assume that the data is
centered. We might have a persistent constant offset vector added to the input
features or the output which will cause the centered recursive least squares
estimator to be inaccurate or have worse generalization. In the static data
regime, estimation of this constant offset can be done prior to solving the
least squares problem by calculating the feature and output means. In online
estimation, we need to not only update the means at each timestep but also to
correct the previous parameter estimate with respect to the new mean estimate.
While the algebra becomes a bit more complex, the eventual structure of the
update equations is remarkably similar to the uncentered case.

In order to contain the complexity of the following derivation, we proceed in
stages. First, we will consider the case when the input features are centered
but the output is not. Then we will consider the inverse case, where the input
features are not centered but the output is. We will then see how these two
cases can be combined in the general uncentered recursive least squares
estimator.

\subsection{Centered X, Uncentered Y}
In this case we want to solve the problem
\begin{equation}
  (y_T - \mu_y(T)) = \varphi_T\hat\Theta(T)
\end{equation}
When the output data we are provided by $F(t)$ are not already centered, we
center it prior to solving the least squares problem. Note that it is simple to calculate the output mean online, since
\begin{align}
  \mu_y(T) &= \frac{1}{T}\sum_{t=1}^T y_t \\
           &= \frac{T - 1}{T}\mu_y(T - 1) + \frac{1}{T}y_T \\
  \implies T\mu_y(T) &= (T - 1)\mu_y(T - 1) + y_T \\
                     &= T \mu_y(T - 1) + y_T - \mu_y(T - 1) \\
  \implies \mu_y(T) &= \mu_y(T - 1) + \frac{1}{T} (y_T - \mu_y(T-1))
\end{align}
Thus the normal equation in this case is given by 
\begin{align}
  (X_T^\top X_T)^{-1} X_T^\top \left(Y_T - \bar{1}\mu_y(T)\right) &= (X_T^\top X_T)^{-1} X_T^\top \left(Y_T - \bar{1}\frac{1}{T}\sum_{t=1}^Ty_t\right) \\
                                                                  &= (X_T^\top X_T)^{-1} X_T^\top Y_T - (X_T^\top X_T)^{-1} X_T^\top \bar{1}\frac{1}{T}\sum_{t=1}^Ty_t \label{eq:split_uncentered_y}
\end{align}
Note that the first term in Equation~\ref{eq:split_uncentered_y} is just the
normal equation for the centered least squares problem that we already derived
recursive update equations for, so all that remains is expanding the second
term as
\begin{align}
  (X_T^\top X_T)^{-1} X_T^\top \bar{1}\mu_y(T) &= P_T \sum_{t=1}^T \varphi_t^\top \mu_y(T) \\
                                               &= T \cdot P_T \mu_x(T)^\top \mu_y(T) \label{eq:correction_1}
\end{align}
Of course, in the current case $\mu_x = \bar{0}$, so this correction term is
actually zero and we recover the same update equations as previously derived.
However, Equation~\ref{eq:correction_1} will come in handy later when we derive
the general uncentered update.

Even though the update equations are the same, the final prediction of our
model includes a constant offset term that we can derive from the model
equation
\begin{align}
  (y_T - \mu_y(T)) &= \varphi_T\hat\Theta_{LS}(T) \\
  \implies y_T &= \varphi_T\hat\Theta_{LS}(T) + \mu_y(T)
\end{align}

\subsection{Uncentered X, Centered Y}
In this case we want to solve the problem
\begin{equation}
  y_T = (\varphi_T - \mu_x(T))\hat\Theta(T)
\end{equation}
In the same way that we calculated the update equation for $\mu_y$, we calculate 
\begin{equation}
  \mu_x(T) = \mu_x(T - 1) + \frac{1}{T}(\varphi_T - \mu_x(T - 1))
\end{equation}
The normal equation for this case is given by
\begin{multline}
  \label{eq:expanded_uncentered_x_normal}
  \left[(X_T - \bar{1}\mu_x(t))^\top(X_T - \bar{1}\mu_x(t))\right]^{-1}(X_T - \bar{1}\mu_x(t))^\top Y = \\ \left[(X_T^\top X_T - X_T^\top \bar{1}\mu_x(T) - (\bar{1}\mu_x(T))^\top X_T + \mu_x(T)^\top\mu_x(T)\right]^{-1}(X_T - \bar{1}\mu_x(t))^\top Y
\end{multline}
Note that
\begin{align}
  X_T^\top \bar{1}\mu_x(T) &= \sum_{t=1}^T \varphi_t^\top \mu_x(T) \\
                           &= T \mu_x(T)^\top \mu_x(T)
\end{align}
Thus Equation~\ref{eq:expanded_uncentered_x_normal} becomes
\begin{multline}
  \left[(X_T^\top X_T - 2T\mu_x(T)^\top\mu_x(T) + \mu_x(T)^\top\mu_x(T)\right]^{-1}(X_T - \bar{1}\mu_x(t))^\top Y = \\ \left[(P_T^{-1} - (2T - 1)\mu_x(T)^\top\mu_x(T)\right]^{-1}(X_T - \bar{1}\mu_x(t))^\top Y
\end{multline}
Let us define, analogously to $P_T$ from before, 
\begin{equation}
  Q_T := \left[P_T^{-1} - (2T - 1)\mu_x(T)^\top\mu_x(T)\right]^{-1}
\end{equation}

\subsubsection{Deriving the $Q_T$ Update}
As with $P_T$, we begin by developing an update for $Q_T^{-1}$ in terms of $Q_{T-1}^{-1}$
\begin{align}
  Q_T^{-1} &= P_T^{-1} - (2T - 1)\mu_x(T)^\top\mu_x(T) \\
           &= P_{T-1}^{-1} + \varphi_T^\top\varphi_T - \frac{2T - 1}{T^2}((T-1)\mu_x(T-1) + \varphi_T)^\top((T-1)\mu_x(T-1) + \varphi_T) \label{eq:Q_T_partial}
\end{align}
As a side computation, and to avoid stacking even longer equations, let $\mu := \mu_x(T - 1)$ and note
\begin{align}
  ((T - 1)\mu + \varphi_T)^\top((T-1)\mu + \varphi_T) &= (T - 1)^2\mu^\top\mu + (T-1)\mu^\top\varphi_T + (T-1)\varphi_T^\top\mu + \varphi_T^\top\varphi_T
\end{align}
With this, Equation~\ref{eq:Q_T_partial} can be written as
\begin{align}
  Q_T^{-1} &= (P_{T-1}^{-1} - (2T - 1)\mu^\top\mu + 2\mu^\top\mu) - 2\mu^\top\mu + \varphi_T^\top\varphi_T- \frac{2T - 1}{T^2}\left[(-2T + 1)\mu^\top\mu + (T-1)\mu^\top\varphi_T + (T-1)\varphi_T^\top\mu + \varphi_T^\top\varphi_T\right] \\
           &= (P_{T-1}^{-1} - (2(T-1) - 1)\mu^\top\mu) - 2\mu^\top\mu + \varphi_T^\top\varphi_T + \frac{2T - 1}{T^2}\left[(-2T + 1)\mu^\top\mu + (T-1)\mu^\top\varphi_T + (T-1)\varphi_T^\top\mu + \varphi_T^\top\varphi_T\right] \\
           &= Q_{T-1}^{-1} - 2\mu^\top\mu + \varphi_T^\top\varphi_T- \frac{2T - 1}{T^2}\left[(-2T + 1)\mu^\top\mu + (T-1)\mu^\top\varphi_T + (T-1)\varphi_T^\top\mu + \varphi_T^\top\varphi_T\right]
\end{align}
One final expansion:
\begin{equation}
  Q_T^{-1} = Q_{T-1}^{-1} + \frac{1}{T^2}\left[((2T+1)^2 - 2T^2)\mu^\top\mu - (2T - 1)(T - 1)\mu^\top\varphi_T - (2T - 1)(T - 1)\varphi_T^\top\mu + (T^2 - 2T + 1)\varphi_T^\top\varphi_T\right]
\end{equation}
And now we can see that this can be written as
\begin{equation}
  Q_T^{-1} = Q_{T-1}^{-1} + \frac{1}{T^2}
  \begin{bmatrix}
    \mu_x(T-1)^\top & \varphi_T^\top
  \end{bmatrix}
  \begin{bmatrix}
    (2T - 1)^2 - 2T^2 & -(2T - 1)(T - 1) \\
    -(2T - 1)(T - 1) & (T - 1)^2
  \end{bmatrix}
  \begin{bmatrix}
    \mu_x(T - 1) \\
    \varphi_T
  \end{bmatrix}
\end{equation}
Let us define
\begin{align}
  C_T &:= \frac{1}{T^2}
  \begin{bmatrix}
    (2T - 1)^2 - 2T^2 & -(2T - 1)(T - 1) \\
    -(2T - 1)(T - 1) & (T - 1)^2
  \end{bmatrix} \\
  V_T &:= 
  \begin{bmatrix}
    \mu_x(T - 1) \\
    \varphi_T
  \end{bmatrix} \\
  R_T &:= V_T^\top C_T V_T
\end{align}
So that 
\begin{equation}
  \label{eq:Q_T_inv}
  Q_T^{-1} = Q_{T - 1}^{-1} + R_T
\end{equation}
The Woodbury matrix identity (Equation~\ref{eq:woodbury}) gives us, at the end
of all this, an update rule for $Q_T$:
\begin{equation}
  Q_T = Q_{T - 1} - Q_{T - 1}V_T^\top\left(C_T^{-1} + V_TQ_{T-1}V_T^\top\right)^{-1}V_TQ_{T-1} \label{eq:uncentered_Q_T_update}
\end{equation}
Note that this is a rank-2 update to $Q_{T-1}$, since we are using both the
sample mean at time $T - 1$ and the new data at time $T$ to compute the update.

\subsubsection{Deriving the $\hat\Theta$ Update}
Returning to the normal equation
\begin{align}
  \hat\Theta_{LS}(T) &= Q_T (X_T - \bar{1} \mu_x(T))^\top Y \\
                     &= Q_TX^\top Y - Q_T \mu_x(T)^\top\bar{1}^\top Y \\
                     &= Q_T\left[Q_{T-1}^{-1}(t-1)\hat\Theta_{LS}(T-1) + \varphi_T^\top y_T\right] - TQ_T\mu_x(T)^\top \mu_y(T) \\
                     &= Q_T \varphi_T^\top y_T + Q_T\left[Q_{T}^{-1} - R_T\right]\hat\Theta_{LS}(T-1) - TQ_T\mu_x(T)^\top\mu_y(T) \\
                     &= \hat\Theta_{LS}(T-1) + Q_T\left[\varphi_T^\top y_T - R_T\hat\Theta_{LS}(T-1)\right] - TQ_T\mu_x(T)^\top \mu_y(T) \label{eq:uncentered_x_theta_with_correction}
\end{align}
Since in the current case we are assuming that $Y$ is already centered,
$\mu_y(T) = \bar{0}$ and the above reduces to
\begin{equation}
  \hat\Theta_{LS}(T) = \hat\Theta_{LS}(T-1) + Q_T\left[\varphi_T^\top y_T - R_T\hat\Theta_{LS}(T-1)\right] \label{eq:uncentered_x_theta_update}
\end{equation}
This corresponds roughly to the $\hat\Theta$ update in the centered case (Equation~\ref{eq:centered_theta_update}).

\subsection{Uncentered X, Uncentered Y}
We have finally arrived at the general case, in which our problem is expressed as 
\begin{equation}
  (y_T - \mu_y(T)) = (\varphi_T - \mu_x(T))\hat\Theta(T)
\end{equation}
The normal equation for this case is 
\begin{align}
  \left[(X_t - \bar{1}\mu_x(T))^\top(X_T - \bar{1}\mu_x(T))\right]^{-1}(X_T &- \bar{1}\mu_x(T))^\top (Y_T - \bar{1}\mu_y(T)) \nonumber \\ &= Q_T(X_T - \bar{1}\mu_x(T))^\top (Y_T - \bar{1}\mu_y(T)) \\
                                                                            &= Q_T (X_T - \bar{1}\mu_x(T))^\top Y - Q_T(X_T - \bar{1}\mu_x(T))^\top\bar{1}\mu_y(T) \label{eq:uncentered_xy_normal}
\end{align}
Note that the first term in Equation~\ref{eq:uncentered_xy_normal} is the
normal equation for the uncentered X, centered Y case previously analyzed. Thus
all that we need to do in order to derive the update for $\hat\Theta$ is expand the second term. This is easy, though, because
\begin{align}
  Q_T(X_T - \bar{1}\mu_x(T))^\top\bar{1}\mu_y(T) &= Q_TX_T^\top\bar{1}\mu_y(T) - Q_T\mu_x(T)^\top\bar{1}^\top\bar{1}\mu_y(T)\\
                                                 &= TQ_T\mu_x(T)^\top \mu_y(T) - Q_T\mu_x(T)^\top\mu_y(T) \\
                                                 &= (T - 1)Q_T \mu_x(T)^\top \mu_y(T)
\end{align}
This, combined with the correction term in
Equation~\ref{eq:uncentered_x_theta_with_correction} (which is now nonzero
since we assume that $\mu_Y(T) \neq 0$), gives us a total correction of $(2T -
1) Q_T \mu_x(T)^\top \mu_y(T)$.

Since the update equation for $Q_T$ depends only on $\varphi_T$ and $\mu_x(T)$,
it only remains to give the final update equation for $\hat\Theta(T)$ in the
uncentered X, uncentered Y case. Combining Equation~\ref{eq:uncentered_x_theta_update} with the previously stated correction term, we get the following update:
\begin{align}
  \hat\Theta_{RAW}(T) &= \hat\Theta_{RAW}(T-1) + Q_T\left[\varphi_T^\top y_T - R_T\hat\Theta_{RAW}(T-1)\right]\\
  \hat\Theta_{LS}(T) &= \hat\Theta_{RAW}(T) - (2T - 1)Q_T\mu_x(T)^\top \mu_y(T)
\end{align}

Recall that our original problem in the uncentered case was
\begin{equation}
  (y_T - \mu_y(T)) = (\varphi_T - \mu_x(T))\hat\Theta
\end{equation}
Thus the prediction of the recursive least squares filter in the uncentered X, uncentered Y case for a new $\varphi'$ is given by
\begin{align}
  (\hat{y}' - \mu_y(T)) &= (\varphi' - \mu_x(T))\hat\Theta_{LS}(T) \\
  \implies \hat{y}' &= \varphi'\hat\Theta_{LS}(T) + (\mu_y(T) - \mu_x(T)\hat\Theta_{LS}(T))
\end{align}


\section{Practical Issues}
\label{sec:practical}
\subsection{Why Can't We Just Modify the Feature Vector?}
In some data science and machine learning circles, a common ad hoc way to avoid centering data prior to solving the least squares problem is to append a constant dimension to the feature vector, such that the new feature vectors are given by 
\begin{equation}
  \lambda'_t = 
  \begin{bmatrix}
    \lambda_t & 1
  \end{bmatrix}
\end{equation}
Intuitively, the intention of this technique is to add an additional offset
parameter $\theta_0$ to the least squares estimated parameters, which should
then capture the constant offset term $\mu_y - \mu_x \hat\Theta$ which we
calculated analytically above. In practice, though, this technique gives rise
to an entire one-dimensional subspace of possible solutions to the
least-squares problem, as can be shown using a bit of linear algebra, where the
actual solution returned by solving the normal equations is determined by the
particular numerical algorithm used to invert the covariance matrix. The
addition of a constant feature to all input data makes the sample covariance
matrix $X_T^\top X_T$ low-rank, so that the inverse $(X_T^\top X_T)^{-1}$ is
ill-posed and gives rise to a subspace of possible solutions. The best solution
of this subspace in expectation is precisely the one derived in
Section~\ref{sec:uncentered}, but the feature vector augmentation technique
gives no guarantees of recovering this solution.

\subsection{Initializing $P_T$ and $Q_T$ and Connections to Ridge Regression}
The update equations derived in Sections~\ref{sec:centered}
and~\ref{sec:uncentered} tell use how to move from the least-squares solution
at timestep $T-1$ to the least-squares solution at timestep $T$, but they don't
tell us how the relevant matrices should be initialized. For all involved
matrices except $P_T$ (or $Q_T$, in the uncentered case), it is reasonable to
initialize with matrices whose elements are all 0. If we initialize $P_T$ or
$Q_T$ to zero matrices, however, Equations~\ref{eq:centered_p_update} and~\ref{eq:uncentered_Q_T_update} show that these matrices will never be updated at all (since the update equations are multiplicative in $P_T$ and $Q_T$, respectively). 

This means that we need to initialize $P_T$ and $Q_T$ to some nonzero matrix
before beginning the recursive least squares algorithm. In practice, this
initialization matrix is usually chosen as some multiple of the identity, so
that $P_0 = \alpha I$. This has a significant effect on the computed $\hat\Theta$, however, as can be seen if we examine the real normal equation in this situation:
\begin{equation}
  \hat\Theta_{REAL}(T) = (X_T^\top X_T + \frac{1}{\alpha} I)^{-1}X_T^\top Y
\end{equation}
This is precisely the solution for the ridge regression problem (also known as
$l^2$ regularized least squares or Tikhonov regularization) with regularization
coefficient $\frac{1}{\alpha}$. Thus any practical implementation of recursive
least squares which keeps around an estimate of the inverse covariance matrix
($P_T$ or $Q_T$ in our notation) is in fact computing a ridge regression
estimator. This explains the common advice to use $\alpha \approx 10^6$; a
large value for $\alpha$ corresponds to a low amount of $l^2$ regularization,
and hence closer approximation to the unregularized least squares solution.

\subsection{Forgetting Factors for Time-Varying Systems}
When the relationship between $\varphi_t$ and $y_t$ is assumed to change over
time, we need some way of prioritizing recent data over historical data in
recursive least squares. This is commonly done via a ``forgetting factor''
$\lambda \in [0, 1]$. $\lambda$ is used to give exponentially smaller weight to
older samples in the regression in a way that can be intuitively explained by
its extremal values: when $\lambda = 0$ no datum prior to the current timestep
is taken into account, and when $\lambda = 1$ we recover the recursive least
squares algorithm derived above. Common values of $\lambda$ lie between $0.95$
and $0.99$.

The way that this is practically done is by reweighting the rows of the data matrix $X$ and target matrix $Y$. Where previously these were defined as simply the vertically stacked samples, we now define them as 
\begin{align}
  &X = 
  \begin{bmatrix}
    \varphi_T \\
    \lambda \varphi_{T-1} \\
    \vdots \\
    \lambda^{T - 2} \varphi_2 \\
    \lambda^{T - 1} \varphi_1
  \end{bmatrix}
  &Y = 
  \begin{bmatrix}
    y_T \\
    \lambda y_{T-1} \\
    \vdots \\
    \lambda^{T - 2} y_2 \\
    \lambda^{T - 1} y_1
  \end{bmatrix}
\end{align}
Similarly we redefine $\mu_x(T)$ and $\mu_y(T)$ to be the means of these new matrices:
\begin{align}
  &\mu_x(T) = \frac{1}{T}\sum_{t=1}^T \lambda^{T - t} \varphi_t
  &\mu_y(T) = \frac{1}{T}\sum_{t=1}^T \lambda^{T - t} y_t
\end{align}
By carrying this new $X$ and $Y$ through the same derivation as in the uncentered $X$, uncentered $Y$ case above, we can derive analogous matrices and update rules:
\begin{align}
  &C_T := \frac{1}{\lambda^2T^2}
  \begin{bmatrix}
    (2T - 1)^2 - 2T^2 & -(2T - 1)(T - 1) \\
    -(2T - 1)(T - 1) & (T - 1)^2
  \end{bmatrix} 
  &V_T := 
  \begin{bmatrix}
    \lambda\mu_x(T - 1) \\
    \varphi_T
  \end{bmatrix} \\
  &R_T := V_T^\top C_T V_T
\end{align}
\begin{equation}
  Q_T = \frac{1}{\lambda^2} \left[Q_{T-1} - Q_{T-1}V_T^\top\left(C_T^{-1} + V_TQ_{T-1}V_T^\top\right)^{-1}V_TQ_{T-1}\right]
\end{equation}
\begin{align}
  \hat\Theta_{RAW}(T) &= \hat\Theta_{RAW}(T-1) + Q_T\left[\varphi_T^\top y_T - R_T\hat\Theta_{RAW}(T-1)\right]\\
  \hat\Theta_{LS}(T) &= \hat\Theta_{RAW}(T) - (2T - 1)Q_T\mu_x(T)^\top \mu_y(T)
\end{align}

One final note is in order about the recursive least squares algorithm with
$\lambda$-forgetting: The connection that we drew earlier between the
initialization of $Q_T$ and $l^2$ regularized least squares no longer holds, as
the initial setting of $Q_T$ is multiplied by $\lambda^2$ at each timestep, and so
the equivalent regularization coefficient gets smaller with each new
datapoint. More precisely, the recursive least squares solution with
$\lambda$-forgetting and an initialization of $Q_0 = \alpha I$ will, at time
$T$, compute the equivalent ridge regression solution
\begin{equation}
  \hat\Theta_{REAL} = (X_T^\top X_T + \frac{\lambda^{2T}}{\alpha}I)^{-1}X_T^\top Y
\end{equation}

This may be desirable if one wishes to smoothly interpolate between
the ridge regression solution when little data is available and the
unregularized least squares solution in the limit of infinite data, but we are
not aware of any existing statistical analysis of this interpolation.


\newpage

\bibliographystyle{unsrt}
\bibliography{bib/main}

\end{document}
