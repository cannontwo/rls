From Equation~\ref{eq:theta_ls} we can begin to derive the recursive least
squares estimator. It is worth noting at the beginning of this derivation that
we have implicitly assumed that our data is already centered; in other words,
the relationship between $\varphi_t$ and $y_t$ has no offset term, and so our
model $\hat\Theta_{LS}$ will always predict an output vector of all zeros for
an input vector of all zeros. This is a fine assumption when all of the data
has been collected ahead of time, but breaks down when we want to do recursive
least squares because we cannot estimate the means of our inputs and outputs
ahead of time. Since it is easier to derive the recursive least squares
estimator in the centered case than in the uncentered case, we tackle this
limited version first. 

\subsection{Breaking Up the Normal Equation}
We begin by writing out the normal equation as two sums multiplied together
\begin{align}
  \hat\Theta_{LS}(T) &= (X_T^\top X_T)^{-1} X_T^\top Y_T \\
                  &= \left[\sum_{t=1}^T \varphi_t^\top \varphi_t\right]^{-1} \left[\sum_{t=1}^T\varphi_t^\top y_t \right] \label{eq:two_sums}
\end{align}
Let us define the inverse sample covariance matrix $P_T$ to be the left-hand
term in Equation~\ref{eq:two_sums}, so that we then have
\begin{align}
  \label{eq:sample_covar_split}
  P_T^{-1} &= \sum_{t=1}^T \varphi_t^\top \varphi_t \\
           &= P_{T - 1}^{-1} + \varphi_T^\top \varphi_T \label{eq:p_inv_update}\\
  \implies P_{T - 1}^{-1} &= P_T^{-1} - \varphi_T^\top \varphi_T \label{eq:sub_covar}
\end{align}
Similarly, we can break up the right-hand term in Equation~\ref{eq:two_sums}:
\begin{equation}
  \sum_{t=1}^T\varphi_t^\top y_t = \sum_{t=1}^{T-1} \varphi_t^\top y_t + \varphi_T^\top y_T
\end{equation}

\subsection{Deriving the $\hat\Theta$ Update}
Our normal equation is now
\begin{equation}
  \hat\Theta_{LS} = P_T \cdot \left[\sum_{t=1}^{T-1} \varphi_t^\top y_t + \varphi_T^\top y_T
\right]
\end{equation}
Using the definition of $\hat\Theta_{LS}(T - 1)$, we get
\begin{equation}
  \hat\Theta_{LS}(T) = P_T \cdot \left[P_{T-1}^{-1} \hat\Theta_{LS}(T-1) + \varphi_T^\top y_T\right]
\end{equation}
Substituting in Equation~\ref{eq:sub_covar}:
\begin{align}
  \hat\Theta_{LS}(T) &= P_T \cdot \left[(P_{T}^{-1} - \varphi_T^\top \varphi_T) \hat\Theta_{LS}(T-1) + \varphi_T^\top y_T\right] \\
                     &= \hat\Theta_{LS}(T - 1) - P_T \varphi_T^\top \varphi_T \hat\Theta_{LS}(T-1) + P_T \varphi_T^\top y_T \\
                     &= \hat\Theta_{LS}(T - 1) + P_T \varphi_T^\top \left[y_T - \varphi_T\hat\Theta_{LS}(T-1)\right] \label{eq:centered_theta_update}
\end{align}
Note that the last term in Equation~\ref{eq:centered_theta_update} ($y_T -
\varphi_T\hat\Theta_{LS}(T-1)$) is the prediction error of our model at
timestep $T-1$ on the new datum, so the new estimate of $\Theta$ that we get is
the old estimate plus the prediction error on a new datum filtered by $P_T
\varphi_T^\top$. Intuitively, this represents a reweighting of the prediction
error using our existing estimate of the sample covariance, which effectively
rescales the update to $\Theta$ to take into account the scales of the
coordinates of the features.

\subsection{Deriving the $P_T$ Update}
Though Equation~\ref{eq:p_inv_update} gives us a way to update $P_T^{-1}$
easily with each new datum, to recover $P_T$ and update
$\hat\theta_{LS}$ we would need to invert an $n\times n$ matrix at on each time
step. This is not only computationally expensive for all but small values of
$n$, it also introduces the risk of running into floating-point errors if
$P_T^{-1}$ ever becomes ill-conditioned\footnote{These sorts of issues can also
be dealt with using any number of techniques from numerical linear
algebra~\cite{trefethen1997numerical}}.

We can get around these issues by doing away with $P_T^{-1}$ all together and
deriving a direct update for $P_T$. We do this with the Woodbury matrix
identity~\cite{woodbury1950inverting}, also known as the ``matrix inversion
lemma.'' This result tells us that for matrices $A, U, C, V$ such that $UCV$ has rank $k$, the inverse of the rank-$k$ update is given by:
\begin{equation}
  \label{eq:woodbury}
  \left(A + UCV\right)^{-1} = A^{-1} - A^{-1}U\left(C^{-1} + VA^{-1}U\right)^{-1}VA^{-1}
\end{equation}
In our case, since we are doing a rank-1 update $\varphi_T^\top \varphi_T$ to
$P_{T-1}^{-1}$, this lemma gives us that
\begin{equation}
  \label{eq:centered_p_update}
  \left(P_{T-1}^{-1} + \varphi_T^\top \varphi_T\right)^{-1} = P_{T - 1} - \frac{P_{T-1}\varphi_T^\top \varphi_T P_{T-1}}{1 + \varphi_T P_{T-1} \varphi_T^\top}
\end{equation}
And just like that, we're done! Equations~\ref{eq:centered_theta_update}
and~\ref{eq:centered_p_update} give us the update equations that define the
recursive least squares algorithm. At each timestep $t$, we simply need to:
\begin{enumerate}
\begin{singlespace}
  \item Record $\varphi_t$ and $y_t$ from our datastream $F(t)$. 
  \item Calculate $P_t$ from $P_{t-1}$ and $\varphi_t$ using Equation~\ref{eq:centered_p_update}. 
  \item Calculate $\hat\Theta_{LS}(t)$ from $\hat\Theta_{LS}(t-1)$, $\varphi_t$, $y_t$, and $P_{t}$ using Equation~\ref{eq:centered_theta_update}. 
\end{singlespace}
\end{enumerate}
