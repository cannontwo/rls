In the general linear regression setting, we cannot assume that the data is
centered. We might have a persistent constant offset vector added to the input
features or the output which will cause the centered recursive least squares
estimator to be inaccurate or have worse generalization. In the static data
regime, estimation of this constant offset can be done prior to solving the
least squares problem by calculating the feature and output means. In online
estimation, we need to not only update the means at each timestep but also to
correct the previous parameter estimate with respect to the new mean estimate.
While the algebra becomes a bit more complex, the eventual structure of the
update equations is remarkably similar to the uncentered case.

In order to contain the complexity of the following derivation, we proceed in
stages. First, we will consider the case when the input features are centered
but the output is not. Then we will consider the inverse case, where the input
features are not centered but the output is. We will then see how these two
cases can be combined in the general uncentered recursive least squares
estimator.

\subsection{Centered X, Uncentered Y}
In this case we want to solve the problem
\begin{equation}
  (y_T - \mu_y(T)) = \varphi_T\hat\Theta(T)
\end{equation}
When the output data we are provided by $F(t)$ are not already centered, we
center it prior to solving the least squares problem. Note that it is simple to calculate the output mean online, since
\begin{align}
  \mu_y(T) &= \frac{1}{T}\sum_{t=1}^T y_t \\
           &= \frac{T - 1}{T}\mu_y(T - 1) + \frac{1}{T}y_T \\
  \implies T\mu_y(T) &= (T - 1)\mu_y(T - 1) + y_T \\
                     &= T \mu_y(T - 1) + y_T - \mu_y(T - 1) \\
  \implies \mu_y(T) &= \mu_y(T - 1) + \frac{1}{T} (y_T - \mu_y(T-1))
\end{align}
Thus the normal equation in this case is given by 
\begin{align}
  (X_T^\top X_T)^{-1} X_T^\top \left(Y_T - \bar{1}\mu_y(T)\right) &= (X_T^\top X_T)^{-1} X_T^\top \left(Y_T - \bar{1}\frac{1}{T}\sum_{t=1}^Ty_t\right) \\
                                                                  &= (X_T^\top X_T)^{-1} X_T^\top Y_T - (X_T^\top X_T)^{-1} X_T^\top \bar{1}\frac{1}{T}\sum_{t=1}^Ty_t \label{eq:split_uncentered_y}
\end{align}
Note that the first term in Equation~\ref{eq:split_uncentered_y} is just the
normal equation for the centered least squares problem that we already derived
recursive update equations for, so all that remains is expanding the second
term as
\begin{align}
  (X_T^\top X_T)^{-1} X_T^\top \bar{1}\mu_y(T) &= P_T \sum_{t=1}^T \varphi_t^\top \mu_y(T) \\
                                               &= T \cdot P_T \mu_x(T)^\top \mu_y(T) \label{eq:correction_1}
\end{align}
Of course, in the current case $\mu_x = \bar{0}$, so this correction term is
actually zero and we recover the same update equations as previously derived.
However, Equation~\ref{eq:correction_1} will come in handy later when we derive
the general uncentered update.

Even though the update equations are the same, the final prediction of our
model includes a constant offset term that we can derive from the model
equation
\begin{align}
  (y_T - \mu_y(T)) &= \varphi_T\hat\Theta_{LS}(T) \\
  \implies y_T &= \varphi_T\hat\Theta_{LS}(T) + \mu_y(T)
\end{align}

\subsection{Uncentered X, Centered Y}
In this case we want to solve the problem
\begin{equation}
  y_T = (\varphi_T - \mu_x(T))\hat\Theta(T)
\end{equation}
In the same way that we calculated the update equation for $\mu_y$, we calculate 
\begin{equation}
  \mu_x(T) = \mu_x(T - 1) + \frac{1}{T}(\varphi_T - \mu_x(T - 1))
\end{equation}
The normal equation for this case is given by
\begin{multline}
  \label{eq:expanded_uncentered_x_normal}
  \left[(X_T - \bar{1}\mu_x(t))^\top(X_T - \bar{1}\mu_x(t))\right]^{-1}(X_T - \bar{1}\mu_x(t))^\top Y = \\ \left[(X_T^\top X_T - X_T^\top \bar{1}\mu_x(T) - (\bar{1}\mu_x(T))^\top X_T + \mu_x(T)^\top\mu_x(T)\right]^{-1}(X_T - \bar{1}\mu_x(t))^\top Y
\end{multline}
Note that
\begin{align}
  X_T^\top \bar{1}\mu_x(T) &= \sum_{t=1}^T \varphi_t^\top \mu_x(T) \\
                           &= T \mu_x(T)^\top \mu_x(T)
\end{align}
Thus Equation~\ref{eq:expanded_uncentered_x_normal} becomes
\begin{multline}
  \left[(X_T^\top X_T - 2T\mu_x(T)^\top\mu_x(T) + \mu_x(T)^\top\mu_x(T)\right]^{-1}(X_T - \bar{1}\mu_x(t))^\top Y = \\ \left[(P_T^{-1} - (2T - 1)\mu_x(T)^\top\mu_x(T)\right]^{-1}(X_T - \bar{1}\mu_x(t))^\top Y
\end{multline}
Let us define, analogously to $P_T$ from before, 
\begin{equation}
  Q_T := \left[P_T^{-1} - (2T - 1)\mu_x(T)^\top\mu_x(T)\right]^{-1}
\end{equation}

\subsubsection{Deriving the $Q_T$ Update}
As with $P_T$, we begin by developing an update for $Q_T^{-1}$ in terms of $Q_{T-1}^{-1}$
\begin{align}
  Q_T^{-1} &= P_T^{-1} - (2T - 1)\mu_x(T)^\top\mu_x(T) \\
           &= P_{T-1}^{-1} + \varphi_T^\top\varphi_T - \frac{2T - 1}{T^2}((T-1)\mu_x(T-1) + \varphi_T)^\top((T-1)\mu_x(T-1) + \varphi_T) \label{eq:Q_T_partial}
\end{align}
As a side computation, and to avoid stacking even longer equations, let $\mu := \mu_x(T - 1)$ and note
\begin{align}
  ((T - 1)\mu + \varphi_T)^\top((T-1)\mu + \varphi_T) &= (T - 1)^2\mu^\top\mu + (T-1)\mu^\top\varphi_T + (T-1)\varphi_T^\top\mu + \varphi_T^\top\varphi_T
\end{align}
With this, Equation~\ref{eq:Q_T_partial} can be written as
\begin{align}
  Q_T^{-1} &= (P_{T-1}^{-1} - (2T - 1)\mu^\top\mu + 2\mu^\top\mu) - 2\mu^\top\mu + \varphi_T^\top\varphi_T- \frac{2T - 1}{T^2}\left[(-2T + 1)\mu^\top\mu + (T-1)\mu^\top\varphi_T + (T-1)\varphi_T^\top\mu + \varphi_T^\top\varphi_T\right] \\
           &= (P_{T-1}^{-1} - (2(T-1) - 1)\mu^\top\mu) - 2\mu^\top\mu + \varphi_T^\top\varphi_T + \frac{2T - 1}{T^2}\left[(-2T + 1)\mu^\top\mu + (T-1)\mu^\top\varphi_T + (T-1)\varphi_T^\top\mu + \varphi_T^\top\varphi_T\right] \\
           &= Q_{T-1}^{-1} - 2\mu^\top\mu + \varphi_T^\top\varphi_T- \frac{2T - 1}{T^2}\left[(-2T + 1)\mu^\top\mu + (T-1)\mu^\top\varphi_T + (T-1)\varphi_T^\top\mu + \varphi_T^\top\varphi_T\right]
\end{align}
One final expansion:
\begin{equation}
  Q_T^{-1} = Q_{T-1}^{-1} + \frac{1}{T^2}\left[((2T+1)^2 - 2T^2)\mu^\top\mu - (2T - 1)(T - 1)\mu^\top\varphi_T - (2T - 1)(T - 1)\varphi_T^\top\mu + (T^2 - 2T + 1)\varphi_T^\top\varphi_T\right]
\end{equation}
And now we can see that this can be written as
\begin{equation}
  Q_T^{-1} = Q_{T-1}^{-1} + \frac{1}{T^2}
  \begin{bmatrix}
    \mu_x(T-1)^\top & \varphi_T^\top
  \end{bmatrix}
  \begin{bmatrix}
    (2T - 1)^2 - 2T^2 & -(2T - 1)(T - 1) \\
    -(2T - 1)(T - 1) & (T - 1)^2
  \end{bmatrix}
  \begin{bmatrix}
    \mu_x(T - 1) \\
    \varphi_T
  \end{bmatrix}
\end{equation}
Let us define
\begin{align}
  C_T &:= \frac{1}{T^2}
  \begin{bmatrix}
    (2T - 1)^2 - 2T^2 & -(2T - 1)(T - 1) \\
    -(2T - 1)(T - 1) & (T - 1)^2
  \end{bmatrix} \\
  V_T &:= 
  \begin{bmatrix}
    \mu_x(T - 1) \\
    \varphi_T
  \end{bmatrix} \\
  R_T &:= V_T^\top C_T V_T
\end{align}
So that 
\begin{equation}
  \label{eq:Q_T_inv}
  Q_T^{-1} = Q_{T - 1}^{-1} + R_T
\end{equation}
The Woodbury matrix identity (Equation~\ref{eq:woodbury}) gives us, at the end
of all this, an update rule for $Q_T$:
\begin{equation}
  Q_T = Q_{T - 1} - Q_{T - 1}V_T^\top\left(C_T^{-1} + V_TQ_{T-1}V_T^\top\right)^{-1}V_TQ_{T-1} \label{eq:uncentered_Q_T_update}
\end{equation}
Note that this is a rank-2 update to $Q_{T-1}$, since we are using both the
sample mean at time $T - 1$ and the new data at time $T$ to compute the update.

\subsubsection{Deriving the $\hat\Theta$ Update}
Returning to the normal equation
\begin{align}
  \hat\Theta_{LS}(T) &= Q_T (X_T - \bar{1} \mu_x(T))^\top Y \\
                     &= Q_TX^\top Y - Q_T \mu_x(T)^\top\bar{1}^\top Y \\
                     &= Q_T\left[Q_{T-1}^{-1}(t-1)\hat\Theta_{LS}(T-1) + \varphi_T^\top y_T\right] - TQ_T\mu_x(T)^\top \mu_y(T) \\
                     &= Q_T \varphi_T^\top y_T + Q_T\left[Q_{T}^{-1} - R_T\right]\hat\Theta_{LS}(T-1) - TQ_T\mu_x(T)^\top\mu_y(T) \\
                     &= \hat\Theta_{LS}(T-1) + Q_T\left[\varphi_T^\top y_T - R_T\hat\Theta_{LS}(T-1)\right] - TQ_T\mu_x(T)^\top \mu_y(T) \label{eq:uncentered_x_theta_with_correction}
\end{align}
Since in the current case we are assuming that $Y$ is already centered,
$\mu_y(T) = \bar{0}$ and the above reduces to
\begin{equation}
  \hat\Theta_{LS}(T) = \hat\Theta_{LS}(T-1) + Q_T\left[\varphi_T^\top y_T - R_T\hat\Theta_{LS}(T-1)\right] \label{eq:uncentered_x_theta_update}
\end{equation}
This corresponds roughly to the $\hat\Theta$ update in the centered case (Equation~\ref{eq:centered_theta_update}).

\subsection{Uncentered X, Uncentered Y}
We have finally arrived at the general case, in which our problem is expressed as 
\begin{equation}
  (y_T - \mu_y(T)) = (\varphi_T - \mu_x(T))\hat\Theta(T)
\end{equation}
The normal equation for this case is 
\begin{align}
  \left[(X_t - \bar{1}\mu_x(T))^\top(X_T - \bar{1}\mu_x(T))\right]^{-1}(X_T &- \bar{1}\mu_x(T))^\top (Y_T - \bar{1}\mu_y(T)) \nonumber \\ &= Q_T(X_T - \bar{1}\mu_x(T))^\top (Y_T - \bar{1}\mu_y(T)) \\
                                                                            &= Q_T (X_T - \bar{1}\mu_x(T))^\top Y - Q_T(X_T - \bar{1}\mu_x(T))^\top\bar{1}\mu_y(T) \label{eq:uncentered_xy_normal}
\end{align}
Note that the first term in Equation~\ref{eq:uncentered_xy_normal} is the
normal equation for the uncentered X, centered Y case previously analyzed. Thus
all that we need to do in order to derive the update for $\hat\Theta$ is expand the second term. This is easy, though, because
\begin{align}
  Q_T(X_T - \bar{1}\mu_x(T))^\top\bar{1}\mu_y(T) &= Q_TX_T^\top\bar{1}\mu_y(T) - Q_T\mu_x(T)^\top\bar{1}^\top\bar{1}\mu_y(T)\\
                                                 &= TQ_T\mu_x(T)^\top \mu_y(T) - Q_T\mu_x(T)^\top\mu_y(T) \\
                                                 &= (T - 1)Q_T \mu_x(T)^\top \mu_y(T)
\end{align}
This, combined with the correction term in
Equation~\ref{eq:uncentered_x_theta_with_correction} (which is now nonzero
since we assume that $\mu_Y(T) \neq 0$), gives us a total correction of $(2T -
1) Q_T \mu_x(T)^\top \mu_y(T)$.

Since the update equation for $Q_T$ depends only on $\varphi_T$ and $\mu_x(T)$,
it only remains to give the final update equation for $\hat\Theta(T)$ in the
uncentered X, uncentered Y case. Combining Equation~\ref{eq:uncentered_x_theta_update} with the previously stated correction term, we get the following update:
\begin{align}
  \hat\Theta_{RAW}(T) &= \hat\Theta_{RAW}(T-1) + Q_T\left[\varphi_T^\top y_T - R_T\hat\Theta_{RAW}(T-1)\right]\\
  \hat\Theta_{LS}(T) &= \hat\Theta_{RAW}(T) - (2T - 1)Q_T\mu_x(T)^\top \mu_y(T)
\end{align}

Recall that our original problem in the uncentered case was
\begin{equation}
  (y_T - \mu_y(T)) = (\varphi_T - \mu_x(T))\hat\Theta
\end{equation}
Thus the prediction of the recursive least squares filter in the uncentered X, uncentered Y case for a new $\varphi'$ is given by
\begin{align}
  (\hat{y}' - \mu_y(T)) &= (\varphi' - \mu_x(T))\hat\Theta_{LS}(T) \\
  \implies \hat{y}' &= \varphi'\hat\Theta_{LS}(T) + (\mu_y(T) - \mu_x(T)\hat\Theta_{LS}(T))
\end{align}
